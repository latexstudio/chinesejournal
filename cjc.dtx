% \iffalse meta-comment
%
% Copyright (C) 2019-2019 by Zeping Lee <zepinglee AT gmail.com>
%
% This file may be distributed and/or modified under the
% conditions of the LaTeX Project Public License, either version 1.3c
% of this license or (at your option) any later version.
% The latest version of this license is in
%    https://www.latex-project.org/lppl.txt
% and version 1.3c or later is part of all distributions of LaTeX
% version 2005/12/01 or later.
%
%<*internal>
\iffalse
\fi
\begingroup
  \def\nameoflatex{LaTeX2e}
\expandafter\endgroup\ifx\nameoflatex\fmtname\else
\csname fi\endcsname
%</internal>
%<*install>
\input docstrip.tex
\preamble

Copyright (C) 2019-\the\year by Zeping Lee <zepinglee AT gmail.com>

This file may be distributed and/or modified under the
conditions of the LaTeX Project Public License, either version 1.3c
of this license or (at your option) any later version.
The latest version of this license is in
   https://www.latex-project.org/lppl.txt
and version 1.3c or later is part of all distributions of LaTeX
version 2005/12/01 or later.

\endpreamble
\keepsilent
\askforoverwritefalse
\nopostamble
\generate{
  \file{\jobname.cls}{\from{\jobname.dtx}{class}}
}
\endbatchfile
%</install>
%<*internal>
\fi
%</internal>
%<*driver>
\ProvidesFile{cjc.dtx}
%</driver>
%<class>\NeedsTeXFormat{LaTeX2e}[1999/12/01]
%<class>\ProvidesClass{cjc}
%<*class>
  [2019/06/01 v1.0 A template framework for Chinese dissertations]
%</class>
%
%<*driver>
\documentclass[a4paper]{ltxdoc}
\usepackage[UTF8]{ctex}
\usepackage{amsmath}
\usepackage{unicode-math}
\usepackage{siunitx}
\usepackage{caption}
\usepackage{booktabs}
\usepackage{xcolor}
\usepackage{listings}
\usepackage{hypdoc}

\makeatletter

% 设置字体
\IfFileExists{/System/Library/Fonts/Times.ttc}{
  \setmainfont{Times}
  \setsansfont[Scale=MatchLowercase]{Helvetica}
  \setmonofont[Scale=MatchLowercase]{Menlo}
  \xeCJKsetwidth{‘’“”}{1em}
}{}
\unimathsetup{
  math-style=ISO,
  bold-style=ISO,
}
\setmathfont[
  Extension    = .otf,
  BoldFont     = XITSMath-Bold,
  StylisticSet = 8,
]{XITSMath-Regular}

% 定义一些命令用于写文档
\newcommand\TeXLive{\TeX{} Live}
\newcommand\unicodechar[1]{U+#1(\symbol{"#1})}
\DeclareRobustCommand\file{\nolinkurl}
\DeclareRobustCommand\env{\texttt}
\DeclareRobustCommand\pkg{\textsf}
\DeclareRobustCommand\cls{\textsf}
\DeclareRobustCommand\opt{\texttt}

% 在 doc 的基础上增加 option 的描述
\def\DescribeOption{\leavevmode\@bsphack\begingroup\MakePrivateLetters
  \Describe@Option}
\def\Describe@Option#1{\endgroup
              \marginpar{\raggedleft\PrintDescribeOption{#1}}%
              \SpecialEnvIndex{#1}\@esphack\ignorespaces}
\@ifundefined{PrintDescribeOption}
   {\def\PrintDescribeOption#1{\strut \MacroFont #1\ }}{}

% 调整列表的格式
\setlength\partopsep{\z@}
\def\@listi{\leftmargin\leftmargini
            \parsep \z@
            \topsep 5\p@ \@plus2\p@ \@minus3\p@
            \itemsep2\p@ \@plus\p@ \@minus\p@}
\let\@listI\@listi
\@listi

% listings 的样式
\lstdefinestyle{lstshell}{
  basicstyle      = \small\ttfamily,
  backgroundcolor = \color{lightgray},
  gobble          = 2, % 重要!否则会生成注释符号"%"
  language        = bash,
}
\newcommand\shellcmd[1]{\colorbox{lightgray}{\lstinline[style=lstshell]|#1|}}
\lstnewenvironment{shell}{\lstset{style=lstshell}}{}
\lstnewenvironment{latex}{%
  \lstset{
    basicstyle = \small\ttfamily,
    frame      = single,
    gobble     = 4,
    language   = [LaTeX]TeX,
  }%
}{}

% 调整版本历史和索引的格式
\renewcommand\glossaryname{版本历史}
\GlossaryPrologue{\section*{\glossaryname}}
\def\changes@#1#2#3{%
  \protected@edef\@tempa{%
    \noexpand\glossary{#1 (#2)\levelchar
    \ifx\saved@macroname\@empty
      \space
      \actualchar
    \else
      \saved@indexname
      \actualchar
      \string\verb\quotechar*%
      \verbatimchar\saved@macroname
      \verbatimchar
      : \levelchar
    \fi
    #3}}%
  \@tempa\endgroup\@esphack}
\renewcommand\indexname{命令索引}
\IndexPrologue{%
  \section*{\indexname}
  \textit{意大利体的数字表示描述对应索引项的页码;%
    带下划线的数字表示定义对应索引项的代码行号;%
    罗马字体的数字表示使用对应索引项的代码行号。}%
}

\hypersetup{
  allcolors         = blue,
  bookmarksnumbered = true,
  bookmarksopen     = true,
}
\makeatother

\EnableCrossrefs
\CodelineIndex
\RecordChanges
% \OnlyDescription

\begin{document}
  \DocInput{\jobname.dtx}
  \clearpage
  \linespread{1.3}
  \PrintChanges
  \clearpage
  \linespread{1}
  \PrintIndex
\end{document}
%</driver>
% \fi
%
% \DoNotIndex{\def,\long,\edef,\xdef,\gdef,\let,\global}
% \DoNotIndex{\if,\ifnum,\ifdim,\ifcat,\ifmmode,\ifvmode,\ifhmode,%
%             \iftrue,\iffalse,\ifvoid,\ifx,\ifeof,\ifcase,\else,\or,\fi}
% \DoNotIndex{\begin,\end,\bgroup,\egroup,\begingroup,\endgroup}
% \DoNotIndex{\expandafter,\csname,\endcsname}
% \DoNotIndex{\hsize,\vsize,\hskip,\vskip,\kern,\hfil,\hfill,\hss}
% \DoNotIndex{\hspace,\vspace}
% \DoNotIndex{\p@,\m@ne,\z@,\@ne,\tw@,\@plus,\@minus}
% \DoNotIndex{\newcounter,\setcounter,\addtocounter,}
% \DoNotIndex{\newdim,\newlength,\setlength,\addtolength}
% \DoNotIndex{\newcommand,\renewcommand,\providecommand,\DeclareRobustCommand}
% \DoNotIndex{\newenvironment,\renewenvironment}
% \DoNotIndex{\RequirePackage,\LoadClass,\ProvidesClass}
% \DoNotIndex{\DeclareOption,\CurrentOption,\ExecuteOptions,\ProcessOptions}
% \DoNotIndex{\rmfamily,\sffamily,\ttfamily,\bfseries,\mdseries,\itshape,%
%             \textrm,\textsf,\texttt,\textbf,\textmd,\textit,\textsl,\textsc}
% \DoNotIndex{\iint,\iiint,\iiiint,\oint,\oiint,\oiiint,%
%             \intclockwise,\varointclockwise,\ointctrclockwise,\sumint,%
%             \intbar,\intBar,\fint,\cirfnint,\awint,\rppolint,%
%             \scpolint,\npolint,\pointint,\sqint,\intlarhk,\intx,%
%             \intcap,\intcup,\upint,\lowint}
% \DoNotIndex{\a,\b,\c,\d,\e,\f,\g,\h,\i,\j,\k,\l,%
%             \m,\n,\o,\p,\q,\r,\s,\t,\u,\v,\w,\x,\y,\z,%
%             \A,\B,\C,\D,\E,\F,\G,\H,\I,\J,\K,\L,%
%             \M,\N,\O,\P,\Q,\R,\S,\T,\U,\V,\W,\X,\Y,\Z,%
%             \do\#,\$,\%,\&,\@,\\,\{,\},\^,\_,\~,\ ,\,,\!,\',\",\/,\*,\-}
% \DoNotIndex{\NAT@@close,\NAT@@open,\NAT@cite,\NAT@citenum,\NAT@citesuper,%
%             \NAT@citex,\NAT@citexnum,\NAT@cmt,\NAT@ctype,\NAT@date,%
%             \NAT@last@yr,\NAT@mbox,\NAT@penalty,\NAT@spacechar,%
%             \@citea,\def@NAT@last@yr,\ifNAT@swa}
% \DoNotIndex{\quad,\par,\relax,\ccwd}
% \DoNotIndex{\bp@}
%
%
%
% \GetFileInfo{\jobname.dtx}
%
% \title{\cls{cjc} 使用说明}
% \author{Zeping Lee\thanks{zepinglee AT gmail.com}}
% \date{\filedate\qquad\fileversion}
% \maketitle
%
% \changes{v1.0}{2019/06/01}{Initial version.}
%
%
% \StopEventually
% \clearpage
% \appendix
%
%
% \section{代码实现}
% \label{sec:code}
% \linespread{1}
%
%
% \subsection{处理选项}
%
% 检查编译引擎,要求使用 XeLaTeX。
%    \begin{macrocode}
\RequirePackage{ifxetex}
\RequireXeTeX
%    \end{macrocode}
%
%    \begin{macrocode}
\newif\ifcjc@chinese
\DeclareOption{chinese}{\cjc@chinesetrue}
\DeclareOption{english}{\cjc@chinesefalse}
\DeclareOption*{\PassOptionsToClass{\CurrentOption}{ctexart}}
\ExecuteOptions{chinese}
\ProcessOptions\relax
%    \end{macrocode}
%
%
% \subsection{加载文档类和宏包}
%
%    \begin{macrocode}
\ifcjc@chinese
  \PassOptionsToClass{scheme=chinese}{ctexart}
\else
  \PassOptionsToClass{scheme=plain}{art}
\fi
\PassOptionsToPackage{quiet}{xeCJK}
%    \end{macrocode}
%
% 载入 \cls{ctexart} 文档类,注意要求为 2.2 或更高的版本。
%    \begin{macrocode}
\LoadClass[UTF8,a4paper,twoside,twocolumn,zihao=5,linespread=1.08]{ctexart}[2016/05/16]
%    \end{macrocode}
%
% 检测 ctexart 版本,如果版本过低会报错
%    \begin{macrocode}
\@ifclasslater{ctexart}{2016/05/16}{}{
  \ClassError{cjc}{%
    This template requires TeX Live\MessageBreak 2016 or later version}{}
}
%    \end{macrocode}
%
% 建议在模板开始处载入全部宏包,不要轻易改变加载顺序。
% \pkg{hyperref} 一般在最后加载。
%    \begin{macrocode}
\RequirePackage{amsmath}
\RequirePackage{unicode-math}
\RequirePackage{amsthm}
\RequirePackage{geometry}
\RequirePackage{graphicx}
\RequirePackage{fancyhdr}
\RequirePackage{caption}
\RequirePackage{url}
\RequirePackage[numbers,sort&compress]{natbib}
%    \end{macrocode}
%
%
% \subsection{字体}
%
% \subsubsection{字体库}
%
% \begin{macro}{\cjc@strifeq}
% 使用 \LaTeX3 的功能判断字符串是否相等。这里也可以使用 \pkg{xstring} 宏包。
%    \begin{macrocode}
\newcommand\cjc@strifeq{\csname str_if_eq_x:nnTF\endcsname}
%    \end{macrocode}
% \end{macro}
%
% \begin{macro}{\cjc@fontset}
% 将 \pkg{ctex} 的 \opt{fontset} 存在 \cs{cjc@fontset} 方便 \LaTeXe{} 调用。
%    \begin{macrocode}
\newcommand\cjc@fontset{\csname g__ctex_fontset_tl\endcsname}
%    \end{macrocode}
% \end{macro}
%
% 大部分学位论文都要求西文字体使用 Times New Roman,
% 但是在 Linux 下没有这两个字体,所以使用它们的克隆版 TeX Gyre Termes。
%    \begin{macrocode}
\cjc@strifeq{\cjc@fontset}{fandol}{
  \setmainfont[
    Extension      = .otf,
    UprightFont    = *-regular,
    BoldFont       = *-bold,
    ItalicFont     = *-italic,
    BoldItalicFont = *-bolditalic,
  ]{texgyretermes}
  \setsansfont[
    Extension      = .otf,
    UprightFont    = *-regular,
    BoldFont       = *-bold,
    ItalicFont     = *-italic,
    BoldItalicFont = *-bolditalic,
  ]{texgyreheros}
  \setmonofont[
    Extension      = .otf,
    UprightFont    = *-regular,
    BoldFont       = *-bold,
    ItalicFont     = *-italic,
    BoldItalicFont = *-bolditalic,
    Scale          = MatchLowercase,
  ]{texgyrecursor}
  \ClassWarningNoLine{cjc}{%
    You are using Fandol font family.\MessageBreak
    Some glyphs may be missing.\MessageBreak
    Please switch to a high-quality font set
  }
}{
  \setmainfont{Times New Roman}
  \setsansfont[Scale=MatchLowercase]{Arial}
  \cjc@strifeq{\cjc@fontset}{mac}{
    \setmonofont[Scale=MatchLowercase]{Menlo}
  }{
    \setmonofont[Scale=MatchLowercase]{Courier New}
  }
}
%    \end{macrocode}
%
% 中文字体可以由 \pkg{ctex} 自动设置,但是会有问题:
% \begin{enumerate}
%   \item 无衬线字体默认会使用微软雅黑或者苹方,这对打印不太友好;
%   \item 没有粗体的字体不会开启伪粗;
% \end{enumerate}
% 所以需要重新配置一部分,参考 \pkg{ctex} 宏包。
%    \begin{macrocode}
\cjc@strifeq{\cjc@fontset}{mac}{
  \setCJKmainfont[
       UprightFont = * Light,
          BoldFont = Heiti SC Medium,
        ItalicFont = Kaiti SC,
    BoldItalicFont = Kaiti SC Bold,
  ]{Songti SC}
  \setCJKsansfont[BoldFont=* Medium]{Heiti SC}
  \setCJKfamilyfont{zhsong}[
       UprightFont = * Light,
          BoldFont = * Bold,
  ]{Songti SC}
  \setCJKfamilyfont{zhhei}[BoldFont=* Medium]{Heiti SC}
  \setCJKfamilyfont{zhkai}[BoldFont=* Bold]{Kaiti SC}
  \xeCJKsetwidth{‘’“”}{1em}
}{
  \xeCJKsetup{EmboldenFactor=2}
  \cjc@strifeq{\cjc@fontset}{windows}{
    \IfFileExists{C:/bootfont.bin}{
      \setCJKmainfont[
        BoldFont   = SimHei,
        ItalicFont = KaiTi_GB2312,
      ]{SimSun}
      \setCJKfamilyfont{zhkai}{KaiTi_GB2312}
    }{
      \setCJKmainfont[
        BoldFont   = SimHei,
        ItalicFont = KaiTi
      ]{SimSun}
      \setCJKfamilyfont{zhkai}{KaiTi}
    }
    \setCJKsansfont{SimHei}
    \setCJKfamilyfont{zhsong}{SimSun}
    \setCJKfamilyfont{zhhei}{SimHei}
  }{}
}
%    \end{macrocode}
%
% 使用 \pkg{unicode-math} 配置数学字体。
%    \begin{macrocode}
\unimathsetup{
  math-style = ISO,
  bold-style = ISO,
  nabla      = upright,
  partial    = upright,
}
%    \end{macrocode}
%
% 使用 XITS Math 作为数学字体。
%
% 注意,\cs{IfFontExistsTF} 是 \pkg{fontspec} 2017/01/20 v2.5c 才提供的;
% 而 XITS 字体于 2018-10-03 更改了字体的文件名。
%    \begin{macrocode}
\newif\ifcjc@xitsnew
\@ifpackagelater{fontspec}{2017/01/20}{
  \IfFontExistsTF{XITSMath-Regular.otf}{
    \cjc@xitsnewtrue
  }{}
}{}
\ifcjc@xitsnew
  \setmathfont[
    Extension    = .otf,
    BoldFont     = XITSMath-Bold,
    StylisticSet = 8,
  ]{XITSMath-Regular}
  \setmathfont[range={cal,bfcal},StylisticSet=1]{XITSMath-Regular.otf}
\else
  \setmathfont[
    Extension    = .otf,
    BoldFont     = *bold,
    StylisticSet = 8,
  ]{xits-math}
  \setmathfont[range={cal,bfcal},StylisticSet=1]{xits-math.otf}
\fi
%    \end{macrocode}
%
% \begin{macro}{\cjc@circlefont}
% 五级节标题和脚注需要使用带圈数字,
% 但 Times New Roman 没有这些符号的字形,而华文宋体和中易宋体提供了这些字形,
% 配置在 \cs{cjc@circlefont}。
%    \begin{macrocode}
\cjc@strifeq{\cjc@fontset}{mac}{
  \newfontfamily\cjc@circlefont{Songti SC Light}
}{
  \cjc@strifeq{\cjc@fontset}{windows}{
    \newfontfamily\cjc@circlefont{SimSun}
  }{
    \ifcjc@xitsnew
      \newfontfamily\cjc@circlefont{XITS-Regular.otf}
    \else
      \newfontfamily\cjc@circlefont{xits-regular.otf}
    \fi
  }
}
%    \end{macrocode}
% \end{macro}
%
%
% \subsubsection{字号与行距}
%
% 目前最广泛使用的印刷的长度单位点(磅)通常指 PostScript point
% \footnote{\url{https://en.wikipedia.org/wiki/Point_(typography)}},
% 在 \LaTeX{} 中是 bp,比默认的 pt 略大。
% 这里保存起来可以节约编译时间。
%    \begin{macrocode}
\newdimen\bp@
\bp@=1bp
%    \end{macrocode}
%
%
% \subsection{处理语言}
%
% 由于 Unicode 的一些标点符号是中西文混用的:
% \unicodechar{00B7}、
% \unicodechar{2013}、
% \unicodechar{2014}、
% \unicodechar{2018}、
% \unicodechar{2019}、
% \unicodechar{201C}、
% \unicodechar{201D}、
% \unicodechar{2025}、
% \unicodechar{2026}、
% \unicodechar{2E3A},
% 所以要根据语言设置正确的字体。
% \footnote{\url{https://github.com/CTeX-org/ctex-kit/issues/389}}
% 所以要根据语言设置正确的字体。
%    \begin{macrocode}
\newcommand\cjc@setchinese{%
  \xeCJKResetPunctClass
}
\newcommand\cjc@setenglish{%
  \xeCJKDeclareCharClass{HalfLeft}{"2018, "201C}%
  \xeCJKDeclareCharClass{HalfRight}{%
    "00B7, "2019, "201D, "2013, "2014, "2025, "2026, "2E3A%
  }%
}
\newcommand\cjc@setdefaultlanguage{%
  \ifcjc@chinese
    \cjc@setchinese
  \else
    \cjc@setenglish
  \fi
}
%    \end{macrocode}
%
%
% \subsection{纸张和页面}
% 使用 \pkg{geometry} 宏包设置纸张和页面。
%
% 纸张:A4;
%
% 页面设置:上、下 2.6 cm,左、右 2 cm,页眉 1.7 cm,页脚 1.7 cm。
%
% 注意这里指的是页眉顶部到纸张顶部的距离为 1.7 cm,
% 所以 $\text{footskip} = \SI{2.6}{cm} - \SI{1.7}{cm} = \SI{0.9}{cm}$
%
%    \begin{macrocode}
\geometry{
  paper      = a4paper,
  vmargin    = 2.6cm,
  hmargin    = 2cm,
  headheight = 0.75cm,
  headsep    = 0.15cm,
  twocolumn  = true,
  columnsep  = 0.75cm,
}
%    \end{macrocode}
%
% 使用 \pkg{fancy} 需要先调用 |\pagestyle{fancy}| 再修改 \cs{sectionmark}。
%    \begin{macrocode}
\pagestyle{fancy}
\let\sectionmark\@gobble
\renewcommand\headrulewidth{0.4\p@}
%    \end{macrocode}
%
% 页眉与该部分的章标题相同,宋体 10.5 磅(五号)居中。
% 页码:宋体 10.5 磅、页面下脚居中。
%    \begin{macrocode}
\newcommand\cjc@hf@font{\small}
%    \end{macrocode}
%
% 重定义默认的 |plain| page style,会显示页眉和页脚。
%    \begin{macrocode}
\fancypagestyle{main}{%
  \fancyhf{}%
  \fancyhead[LO]{\cjc@hf@font ? 期}%
  \fancyhead[CO]{\cjc@hf@font 作者名等:论文题目}%
  \fancyhead[RO,LE]{\cjc@hf@font\thepage}%
  \fancyhead[CE]{\cjc@hf@font 计算机学报}%
  \fancyhead[RE]{\cjc@hf@font \the\year{} 年}%
}
\pagestyle{main}
%    \end{macrocode}
%
% |headings| 只有页眉,没有页脚,用于研究生的符号说明和本科生的 front matter。
%    \begin{macrocode}
\fancypagestyle{plain}{%
  \fancyhf{}%
  \fancyhead[L]{\cjc@hf@font 第 ?? 卷\quad 第 ? 期\\\the\year{} 年 \the\month{} 月}%
  \fancyhead[C]{\cjc@hf@font 计算机学报\\CHINESE JOURNAL OF COMPUTERS}%
  \fancyhead[R]{\cjc@hf@font Vol. ??  No. ?\\???.20??}%
}
%    \end{macrocode}
%
%
% \subsection{标题}
%
% \begin{macro}{\maketitle}
% 定义命令用于录入信息。
%    \begin{macrocode}
\def\@maketitle{%
  \newpage
  \null
  \vskip 2em%
  \begin{center}%
  \let \footnote \thanks
    % {\LARGE \@title \par}%
    {\heiti\zihao{2}\@title \par}%
    \vskip 1.5em%
    % {\large
    {\zihao{3}%
      \lineskip .5em%
      \begin{tabular}[t]{c}%
        \@author
      \end{tabular}\par}%
    \vskip 1em%
    % {\large \@date}%
  \end{center}%
  \par
  \vskip 1.5em}
%    \end{macrocode}
% \end{macro}
%
%
% \subsection{摘要}
%
% \begin{environment}{abstract}
%    \begin{macrocode}
\renewenvironment{abstract}{%
    \if@twocolumn
      \section*{\abstractname}%
    \else
      \small
      \begin{center}%
        {\bfseries \abstractname\vspace{-.5em}\vspace{\z@}}%
      \end{center}%
      \quotation
    \fi}
    {\if@twocolumn\else\endquotation\fi}
%    \end{macrocode}
% \end{environment}
%
%
% \subsection{章节标题}
%
% 用 \pkg{ctex} 的接口设置全部章节标题格式。
%
% 一级标题:4 号黑体,段前 8 磅,段后 8 磅。
%    \begin{macrocode}
\ctexset{
  section = {
    format     = \bfseries\zihao{4},
    beforeskip = 8\bp@,
    afterskip  = 8\bp@,
  },
%    \end{macrocode}
%
% 二级标题:5 号黑体,段前 0.25 行,段后 0.25 行。
%    \begin{macrocode}
  subsection = {
    format     = \bfseries\zihao{5},
    beforeskip = 2.625\bp@,
    afterskip  = 2.625\bp@,
  },
%    \end{macrocode}
%
% 三级节标题:5 号宋体,段前段后 0 磅。
%    \begin{macrocode}
  subsubsection = {
    format     = \zihao{5},
    beforeskip = \z@,
    afterskip  = \z@,
  },
}
%    \end{macrocode}
%
%
% \subsection{正文}
%
% \cs{sloppy} 可以减少“overfull boxes”。
%    \begin{macrocode}
\sloppy
%    \end{macrocode}
%
% 禁止扩大段间距。(\href{https://github.com/ustctug/ustcthesis/issues/209}{
%   ustctug/ustcthesis\#209})
%    \begin{macrocode}
\raggedbottom
%    \end{macrocode}
%
% 段间距 0 磅。
%    \begin{macrocode}
\setlength{\parskip}{\z@}
%    \end{macrocode}
%
% URL 的字体设为保持原样。
%    \begin{macrocode}
\urlstyle{same}
%    \end{macrocode}
%
% 使用 \pkg{xurl} 宏包的方法,增加 URL 可断行的位置。
%    \begin{macrocode}
\def\UrlBreaks{%
  \do\/%
  \do\a\do\b\do\c\do\d\do\e\do\f\do\g\do\h\do\i\do\j\do\k\do\l%
     \do\m\do\n\do\o\do\p\do\q\do\r\do\s\do\t\do\u\do\v\do\w\do\x\do\y\do\z%
  \do\A\do\B\do\C\do\D\do\E\do\F\do\G\do\H\do\I\do\J\do\K\do\L%
     \do\M\do\N\do\O\do\P\do\Q\do\R\do\S\do\T\do\U\do\V\do\W\do\X\do\Y\do\Z%
  \do0\do1\do2\do3\do4\do5\do6\do7\do8\do9\do=\do/\do.\do:%
  \do\*\do\-\do\~\do\'\do\"\do\-}
\Urlmuskip=0mu plus 0.1mu
%    \end{macrocode}
%
%
% \subsection{浮动体}
%
% \LaTeX{} 对放置浮动体的要求比较强,这里按照 UK TeX FAQ 的建议
% \footnote{\url{https://texfaq.org/FAQ-floats}} 对其适当放宽。
%    \begin{macrocode}
\renewcommand\topfraction{.85}
\renewcommand\bottomfraction{.7}
\renewcommand\textfraction{.15}
\renewcommand\floatpagefraction{.66}
\renewcommand\dbltopfraction{.66}
\renewcommand\dblfloatpagefraction{.66}
\setcounter{topnumber}{9}
\setcounter{bottomnumber}{9}
\setcounter{totalnumber}{20}
\setcounter{dbltopnumber}{9}
%    \end{macrocode}
%
% 修改默认的浮动体描述符为 |htb|。
%    \begin{macrocode}
\def\fps@figure{htb}
\def\fps@table{htb}
%    \end{macrocode}
%
% 用 \pkg{caption} 宏包设置图、表的格式:
%
% 图片说明字体为小 5 号。
%
% 表号、表题置于表的上方,宋体 10.5 磅居中,单倍行距,段前 6 磅,段后 6 磅,
% 表号与表题文字之间空一字,表号、表题加粗。
% 表注左缩进两字,续行悬挂缩进左对齐,两端对齐。
%    \begin{macrocode}
% \setlength{\floatsep}{6\bp@}
% \setlength{\textfloatsep}{6\bp@}
% \setlength{\intextsep}{6\bp@}
\DeclareCaptionLabelSeparator{zhspace}{\hspace{\ccwd}}
\captionsetup{
  justification  = centerlast,
  font           = small,
  labelsep       = zhspace,
  skip           = 6\bp@,
  figureposition = bottom,
  tableposition  = top,
}
\DeclareCaptionFont{heiti}{\heiti}
\captionsetup[table]{
  font = {heiti,small},
}
%    \end{macrocode}
%
%
% \subsection{数学符号}
%
% 根据中文的数学排印习惯进行设置:
%
% \begin{macro}{\ldots}
% 省略号一律居中,所以 \cs{ldots} 不再居于底部。
%    \begin{macrocode}
\ifcjc@chinese
  \def\mathellipsis{\cdots}
\fi
%    \end{macrocode}
% \end{macro}
%
% \begin{macro}{\le}
% \begin{macro}{\ge}
% 小于等于号、大于等于号要使用倾斜的字形。
%    \begin{macrocode}
\protected\def\le{\leqslant}
\protected\def\ge{\geqslant}
\AtBeginDocument{%
  \renewcommand\leq{\leqslant}%
  \renewcommand\geq{\geqslant}%
}
%    \end{macrocode}
% \end{macro}
% \end{macro}
%
% \begin{macro}{\int}
% 积分号的上下限默认置于上下两侧。
%    \begin{macrocode}
\removenolimits{%
  \int\iint\iiint\iiiint\oint\oiint\oiiint
  \intclockwise\varointclockwise\ointctrclockwise\sumint
  \intbar\intBar\fint\cirfnint\awint\rppolint
  \scpolint\npolint\pointint\sqint\intlarhk\intx
  \intcap\intcup\upint\lowint
}
%    \end{macrocode}
% \end{macro}
%
% \begin{macro}{\Re}
% \begin{macro}{\Im}
% 实部、虚部操作符使用罗马体 $\mathrm{Re}$、$\mathrm{Im}$ 而不是 fraktur 体
% $\Re$、$\Im$。
%    \begin{macrocode}
\AtBeginDocument{%
  \renewcommand\Re{\operatorname{Re}}%
  \renewcommand\Im{\operatorname{Im}}%
}
%    \end{macrocode}
% \end{macro}
% \end{macro}
%
% \begin{macro}{\nabla}
% \cs{nabla} 使用粗正体。
%    \begin{macrocode}
\AtBeginDocument{%
  \renewcommand\nabla{\mbfnabla}%
}
%    \end{macrocode}
% \end{macro}
%
% \begin{macro}{\bm}
% \begin{macro}{\boldsymbol}
% 兼容旧的粗体命令:\pkg{bm} 的 \cs{bm} 和 \pkg{amsmath} 的 \cs{boldsymbol}。
%    \begin{macrocode}
\newcommand\bm{\symbf}
\renewcommand\boldsymbol{\symbf}
%    \end{macrocode}
% \end{macro}
% \end{macro}
%
% \begin{macro}{\square}
% 兼容 \pkg{amssymb} 中的命令。
%    \begin{macrocode}
\newcommand\square{\mdlgwhtsquare}
%    \end{macrocode}
% \end{macro}
%
% 提供一些方便的命令:
%    \begin{macrocode}
\newcommand\upe{\symup{e}}
\newcommand\upi{\symup{i}}
\newcommand\dif{\mathop{}\!\mathrm{d}}
\DeclareMathOperator*{\argmax}{arg\,max}
\DeclareMathOperator*{\argmin}{arg\,min}
%    \end{macrocode}
%
%
% \subsubsection{数学定理}
%
%    \begin{macrocode}
\newtheoremstyle{cjc}
  {\z@}{\z@}
  {}{2\ccwd}
  {\bfseries}{.}
  {\ccwd}{}
\theoremstyle{cjc}
%    \end{macrocode}
% 定义新的定理
%    \begin{macrocode}
\newcommand\cjc@assertionname{断言}
\newcommand\cjc@assumptionname{假设}
\newcommand\cjc@axiomname{公理}
\newcommand\cjc@corollaryname{推论}
\newcommand\cjc@definitionname{定义}
\newcommand\cjc@examplename{例}
\newcommand\cjc@lemmaname{引理}
\newcommand\cjc@proofname{证明}
\newcommand\cjc@propositionname{命题}
\newcommand\cjc@remarkname{注}
\newcommand\cjc@theoremname{定理}
\newtheorem{theorem}             {\cjc@theoremname}
\newtheorem{assertion}  [theorem]{\cjc@assertionname}
\newtheorem{axiom}      [theorem]{\cjc@axiomname}
\newtheorem{corollary}  [theorem]{\cjc@corollaryname}
\newtheorem{lemma}      [theorem]{\cjc@lemmaname}
\newtheorem{proposition}[theorem]{\cjc@propositionname}
\newtheorem{assumption}          {\cjc@assumptionname}
\newtheorem{definition}          {\cjc@definitionname}
\newtheorem{example}             {\cjc@examplename}
\newtheorem*{remark}             {\cjc@remarkname}
%    \end{macrocode}
% \pkg{amsthm} 单独定义了 proof 环境,这里重新定义以满足格式要求。
% 原本模仿 \pkg{amsthm} 写成 |\item[\hskip\labelsep\hskip2\ccwd #1\hskip\ccwd]|,
% 但是却会多出一些间隙。
%    \begin{macrocode}
\renewenvironment{proof}[1][\proofname]{\par
  \pushQED{\qed}%
  \normalfont \topsep6\p@\@plus6\p@\relax
  \trivlist
    \item\relax\hskip2\ccwd
    \textbf{#1}
    \hskip\ccwd\ignorespaces
  }{%
  \popQED\endtrivlist\@endpefalse
}
\renewcommand\proofname\cjc@proofname
%    \end{macrocode}
%
%
% \subsection{参考文献}
%
% \begin{environment}{thebibliography}
% 参考文献列表格式:宋体 10.5 磅,行距 20 磅,续行缩进两字,左对齐。
%    \begin{macrocode}
\renewcommand\bibfont{\fontsize{7.5\bp@}{14\bp@}\selectfont}
\renewcommand\@biblabel[1]{[#1]\hfill}
%    \end{macrocode}
% \end{environment}
%
%
% \subsection{其他宏包的设置}
%
% 这些宏包并非格式要求,但是为了方便同学们使用,在这里进行简单设置。
%    \begin{macrocode}
\newcommand\cjc@atendpackage{\csname ctex_at_end_package:nn\endcsname}
%    \end{macrocode}
%
%
% \subsubsection{\pkg{hyperref} 宏包}
%
%    \begin{macrocode}
\cjc@atendpackage{hyperref}{
  \hypersetup{
    bookmarksnumbered  = true,
    bookmarksopen      = true,
    bookmarksopenlevel = 1,
    linktoc            = all,
    unicode            = true,
    psdextra           = true,
    hidelinks,
  }
%    \end{macrocode}
%
% 填写 PDF 元信息。
%    \begin{macrocode}
  \AtBeginDocument{%
    \hypersetup{
      pdftitle  = \@title,
      pdfauthor = \@author,
    }%
  }
%    \end{macrocode}
%
% 在 PDF 字符串中去掉换行,以减少 \pkg{hyperref} 的警告信息。
%    \begin{macrocode}
  \pdfstringdefDisableCommands{
    \let\\\@empty
    \let\hspace\@gobble
  }
%    \end{macrocode}
%
% \pkg{hyperref} 与 \pkg{unicode-math} 存在一些兼容性问题,见
% \href{https://github.com/ustctug/ustcthesis/issues/223}{
%   ustctug/ustcthesis\#223} 和
% \href{https://github.com/ho-tex/hyperref/pull/90}{ho-tex/hyperref\#90}。
%    \begin{macrocode}
  \@ifpackagelater{hyperref}{2019/04/27}{}{%
    \g@addto@macro\psdmapshortnames{\let\mu\textmugreek}%
  }
%    \end{macrocode}
%
% 设置中文的 \cs{autoref}。
% \footnote{\url{https://tex.stackexchange.com/a/66150/82731}}
%    \begin{macrocode}
  \ifcjc@chinese
    \def\equationautorefname~#1\null{公式~(#1)\null}
    \def\footnoteautorefname{脚注}
    \def\itemautorefname~#1\null{第~#1 项\null}
    \def\figureautorefname{图}
    \def\tableautorefname{表}
    \def\appendixautorefname{附录}
    \def\sectionautorefname~#1\null{第~#1 节\null}
    \def\subsectionautorefname~#1\null{第~#1 节\null}
    \def\subsubsectionautorefname~#1\null{第~#1 节\null}
    \def\theoremautorefname{定理}
    \def\HyRef@autopageref#1{\hyperref[{#1}]{第~\pageref*{#1} 页}}
  \fi
}
%    \end{macrocode}
%
%
% \subsubsection{\pkg{siunitx} 宏包}
%
%    \begin{macrocode}
\cjc@atendpackage{siunitx}{
  \sisetup{
    group-minimum-digits = 4,
    separate-uncertainty = true,
    inter-unit-product   = \ensuremath{{}\cdot{}},
  }
  \ifcjc@chinese
    \sisetup{
      list-final-separator = { 和 },
      list-pair-separator  = { 和 },
      range-phrase         = {~},
    }
  \fi
}
%    \end{macrocode}
%
% \Finale
\endinput
% \Finale
\endinput
